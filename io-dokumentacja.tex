\documentclass[12pt]{article}

\usepackage{geometry}
\usepackage[utf8]{inputenc}
\usepackage[polish]{babel}
\usepackage{polski}
\usepackage{hyperref}
\usepackage{graphicx}
\usepackage{verbatim}
\usepackage{acronym}
\usepackage{fancyhdr}

\hypersetup{
  linkbordercolor={1 1 1},
  urlbordercolor={1 1 1},
  colorlinks=false
}

\pagestyle{fancy}
\cfoot{}
\rfoot{\thepage}

\author{Konrad Delong \and Antek Piechnik}
\title{Inżynieria Oprogramowania - Dokumentacja}

\begin{document}
\maketitle
\tableofcontents
\newpage

\section{Wstep}
Projekt jest realizowany w ramach przedmiotu Inżynieria Oprogramowania. Głównym zadaniem projektu będzie utworzenie systemu typu Named Entity Recognition - wyszukiwanie konkretnego typu słów w tekscie stron WWW (takich jak nazwiska, imiona, czy pseudonimy). Projekt domyślnie ma być integralną częścią systemu SWAT. Sercem systemu ma być algorytm ewolucyjny dopasowujący wykorzystywany zestaw regul językowych do utworzenia najoptymalniejszych kombinacji rozpoznajacych encje danego typu w tekscie. Dzieki takiemu podejsciu nie tylko algorytm wykorzysta podane reguły do stworzenia najbliższej optymalnego stanu kombinacji, ale rowniez pozwoli na latwe dodanie nowych regul do aplikacji.
\section{Struktura projektu}
Projekt bedzie skladal sie z nastepujacych czesci:
\subsection{Tagger tekstow do testow}
Aplikacja majaca na celu skanowanie przykladowych tekstow a nastepnie generowania specyficznych raportow odnosnie ilosci wystapien poszczegolnych encji w tekscie. Wykorzystywane one będą następnie do badania sprawności wykorzystywanego algorytmu ewolucyjnego oraz trafnosci pojawiajacych się nowych reguł.
\subsection{Rdzen}
W tej części aplikacji znajdować się będzie zaimplementowany algorytm który przekazywal będzie zestawy regul, otrzymywal raporty dotyczace ich sprawnosci a następnie tworzyl kolejne zestawy regul na podstaiw otrzymanych wyników. W celu zarówno uproszczenia implementacji jak i ulatwienia dalszej rozbudowy systemu chcemy rozdzelić rdzeń aplikacji od całej reszty i nie powierzać mu zadań takich jak skanowanie tekstu.
\subsection{Skaner - tester}
Aplikacja po otrzymaniu konkretnego zestawu reguł będzie testowała je na paru róznego rodzaju otagowanych wcześniej tekstach a następnie generowała rakorty dla Rdzenia na podstawie których będzie on mógł odpowiednio je modyfikować.
\section{Opis algorytmu}
\subsection{Ogolny opis}
Algorytm ma docelowo być przykładem algorytmu ewolucyjnego który pozwoli na otrzymanie w miarę optymalnych wyników niezależnie od ilości podanych reguł. Zgodnie z przeczytanymi uprzednio artykułami efektywność tego typu rozwiązań sięga nawet 80-90 procent przy odpowiedniej ilości reguł.Algorytm będzie dobierał odpowiednie kombinacje reguł , rozszerzał je lub pomniejszał, zważając na dostępne o pewnego typu kombinacjach informacje (o ile takie będą już istniały), dzięki czemu przy zyciu będą pozostawały przede wszystkim skuteczne kombinacje reguł. Utworzona zostanie również pula słabszych rozwiązań dzięki czemu raz na pewien czas również i te kombinacje będą łączone z innymi, dzięki czemu nie będą one znikały z systemu na zawsze (czasem dwie tego typu reguły mogłyby dać dużo lepsze efekty niż inne w systemie.
\subsection{Zastosowane reguły}
Reguły pomagające rozpoznać nazwisko:
\begin{itemize}
\item Słowo `doktor` poprzedza nazwisko
\item Slowo `mgr` poprzedza nazwiso
\item Slowo `mgr inz.` poprzedz nzwisko
\item Slowa `lek.`
\item Słowo `pan` poprzedza nazwisko
\item Słowo `pani` poprzedza nazwisko
\item Inne tego typu tytuly
\item Imię poprzedza nazwisko
\item Czasownik w trzeciej osobie liczby pojedynczej następuje po nazwisku (czasownik w formie meskiej czy tez dmskiej)
\item Nazwisko kończy się charakterystycznym sufiksem (np. `'-ski`', `'icz`')
\item Nazwisko zaczyna się dużą literą
\item Czasowniki takie jak `zamieszkaly`, 'urodzony`
\end{itemize}
Dzieki zastosowaniu systemu regul oraz ich automatycznej rekombinacji, bez prolemu bedzie mozna do systemu wstawiac dodatkowe reguly co usprawni jego prace oraz uswietni wyniki przez niego osiagane.

\section{Zastosowane technologie}
Do implementacji znacznej czesci aplikacji pragniemy wykorzystac jezyk Python, ze wzgledu na jego perfekcyjnie zastosowania przy prztwarzaniu jezyka naturalnego. Jak do tego pory:
\subsection {Tagger} - Wykonano wstepny projekt taggera w Pythonie na potrzeby testowe.
\subsection {Rdzen} - Rozpoczeto prace  w Pythonie
\section{Opis zastosowanych narzedzi}
\subsection{Jython} - bridge pozwalajacy n transparentna komunikacje miedzy Pythonem a klasami Javy.
\subsection{CLP} - 



\end{document}
