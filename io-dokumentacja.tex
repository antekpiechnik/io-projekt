\documentclass[12pt]{article}

\usepackage{geometry}
\usepackage[utf8]{inputenc}
\usepackage[polish]{babel}
\usepackage{polski}
\usepackage{hyperref}
\usepackage{graphicx}
\usepackage{verbatim}
\usepackage{acronym}
\usepackage{fancyhdr}

\hypersetup{
  linkbordercolor={1 1 1},
  urlbordercolor={1 1 1},
  colorlinks=false
}

\pagestyle{fancy}
\cfoot{}
\rfoot{\thepage}

\author{Konrad Delong \and Antek Piechnik}
\title{Inżynieria Oprogramowania - Dokumentacja}

\begin{document}
\maketitle
\tableofcontents
\newpage

\section{Wstep}
Projekt jest realizowany w ramach przedmiotu Inżynieria Oprogramowania. Głównym zadaniem projektu będzie utworzenie systemu typu Named Entity Recognition - wyszukiwanie konkretnego typu słów w tekscie stron WWW (takich jak nazwiska, imiona, czy pseudonimy). Projekt domyślnie ma być integralną częścią systemu SWAT. Sercem systemu ma być algorytm ewolucyjny dostosowujący 
\section{Struktura projektu}
\section{Opis algorytmu}
Reguły pomagające rozpoznać nazwisko:
\begin{itemize}
\item Słowo ``doktor`` poprzedza nazwisko
\item Słowo ``pan`` poprzedza nazwisko
\item Słowo ``pani`` poprzedza nazwisko
\item Czasowniki takie jak ``został``, ``jest``, ``był`` następują na nazwisku
\item Imię poprzedza nazwisko
\item Czasownik w trzeciej osobie liczby pojedynczej następuje po nazwisku
\item Nazwisko kończy się charakterystycznym sufiksem (np. "-ski")
\item Nazwisko zaczyna się dużą literą
\end{itemize}

\section{Zastosowane technologie}
\section{Opis zastosowanych narzedzi}



\end{document}
